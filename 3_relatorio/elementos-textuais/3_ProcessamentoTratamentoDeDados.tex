\chapter{Processamento e Tratamento dos Dados}

Nesta etapa, nosso principal objetivo foi compatibilizar os diferentes conjuntos de dados, de forma a construir um único conjunto de dados para facilitar o processo de análise e geração dos resultados.

\section{Convertendo todos os dados para um formato de arquivo estruturado}

Os dados obtidos para cada uma das estações meteorológicas convencionais do INMET,  em formato HTML, foram convertidos para o formato CSV utilizando a biblioteca Pandas \cite{mckinney2011pandas} e o kit de ferramentas lxml \cite{behnel2005lxml}. Em seguida, os dados de todas as estações convencionais foram combinados em um único arquivo estruturado em formato CSV. 

Os dados das estações automáticas do INMET já foram obtidos em formato CSV, porém divididos em um arquivo por estação e por ano. Para este conjunto de dados, também utilizamos a biblioteca Pandas para combinar todos os dados em um único arquivo, também no formato CSV.

Os dados das estações automáticas do LabMet foram obtidos em formato XLS, um arquivo por variável para cada estação. Cada arquivo de uma determinada variável, para uma determinada estação, possuia múltiplas planilhas, cada planilha continha os dados para um determinado ano. Utilizamos a biblioteca Pandas para transformar e agrupar todos os dados das diferentes estações, variáveis e anos em um único arquivo estruturado em formato CSV.


Após este processo de transformação, obtivemos os conjuntos de dados descritos na Tabela \ref{tab:lista_conjunto_de_dados}.

\begin{table}[H]
\caption{Conjunto de dados utilizados neste trabalho.}
\label{tab:lista_conjunto_de_dados}
\begin{tabular}{|l|c|}
\hline
\textbf{Nome do Conjunto de Dados} & \textbf{Quantidade de Registros}\\
\hline
Estações Meteorológicas Convencionais - INMET  & 12.251.335 \\
\hline
Estações Meteorológicas Automáticas - INMET & 54.840.384 \\
\hline
Estações Meteorológicas Automáticas - LabMet & 9.892 \\
\hline
\end{tabular}
\end{table}

Antes de avançarmos no processo de tratamento e análise dos dados, preparamos um ambiente apropriado para o processamento desse conjunto de dados. 
Em um computador com 16 núcleos lógicos de processamento, 32GB de memória RAM e 256GB de SSD, configuramos um ambiente docker\footnote{A configuração docker utilizada para criar o ambiente usado no processamento dos dados está disponível em: \href{https://github.com/saraivaufc/bigdata-docker}{https://github.com/saraivaufc/bigdata-docker}.} com as plataformas Hadoop, versão 3.1.2, e Spark, versão 3.0.1. Todos os dados em formato CSV foram inseridos no sistema de arquivos HDFS do Hadoop para serem processados pelo Spark posteriormente.  

O primeiro conjunto de dados, das estações meteorológicas convencionais do INMET, possui de 2 a 3 registros por dia, dependendo do ano observado, enquanto que as estações automáticas do INMET possui 24 registros por dia (horário), e as estações automáticas do LabMet possuem apenas um único registro por dia (diário). Pensando nisso, decidimos transformar todos os dados para a frequência diária. Esta transformação permitiu juntarmos todos os dados em um único grande conjunto com registros diários das variáveis analisadas. Nas próximas seções descreveremos os passos realizados para realizar essa transformação. 

\section{Transformando os dados das estações convencionais do INMET em registros diários}

Ao analisar as variáveis de temperatura máxima, temperatura mínima e temperatura média das estações convencionais do INMET, identificamos em torno de 67\% de registros nulos em cada uma das variáveis. Isso se deve, principalmente, pelo fato dessas variáveis estarem presentes em apenas um dos dois ou três registros diários disponibilizados pelas estações. A temperatura máxima e temperatura média estão presentes no registro das 00:00 horas (UTC) e a temperatura mínima no registro das 12:00 horas (UTC). Com isso, transformamos esse dado para a frequência diária utilizando como registro diário a primeira ocorrência do valor dessas variáveis no dia. Após o procedimento de transformação para dados diários, a quantidade de registro diminuiu de 12.251.335 para 4.224.027 e a porcentagem de dados nulos diminuiu de 67\% para cerca de 6\% nas temperaturas máximas e mínimas, e para próximo de 12\% para a temperatura média. A Tabela \ref{tab:estacoes_convencionais_inmet_dados_nulos} apresenta a quantidade de dados nulos antes e após a transformação para dados diários.  

\begin{table}[H]
\caption{Quantidade de registros nulos antes e após a conversão para dados diários das estações meteorológicas convencionais do INMET.}
\label{tab:estacoes_convencionais_inmet_dados_nulos}
\begin{adjustbox}{width=\textwidth}
\begin{tabular}{|l|c|r|r|r|}
\hline
\textbf{Variável} & \textbf{Registros nulos (Qtd.)} & \textbf{Registros nulos (\%)} & \textbf{Registros nulos diários (Qtd.)} & \textbf{Registros nulos diários (\%)} \\
\hline
Temperatura máxima  & 8.300.413 & 67,75\% & 273.105 & 6,46\% \\
\hline
Temperatura mínima & 8.292.199 & 67,68\% & 264.891 & 6,27\% \\
\hline
Temperatura média & 8.528.574 & 69,61\% & 501.266 & 11,86\% \\
\hline
\end{tabular}
\end{adjustbox}
\end{table}

\section{Transformando os dados das estações automáticas do INMET em registros diários}

Nas estações meteorológicas automáticas do INMET, os valores nulos estão indicados com o valor -9999, então o primeiro passo foi convertermos todos os valores -9999 para o valor "NULO". Em seguida, realizamos a contagem dos valores nulos dos dados horários das variáveis Temperatura Máxima, Temperatura Mínima e Temperatura do Bulbo seco. Utilizamos a temperatura do bulbo seco horária para estimar a temperatura média do ar diária. Esta primeira análise indicou que apenas aproximadamente 4\% dos dados das variáveis analisadas estavam com valores nulos. Os dados estão disponíveis originalmente na periodicidade horária, para convertermos para dados diários de temperatura máxima, mínima e média, calculamos o máximo valor da temperatura máxima, o mínimo valor da temperatura mínima e a média dos 24 valores da temperatura do bulbo seco, respectivamente. Após a conversão dos dados horários para dados diários, a quantidade de registros passaram de 54.840.384 para 2.072.346, e a quantidade de dados nulos passaram de aproximadamente 4\% para cerca de 6\% nos dados diários. O aumento em proporção dos dados nulos na escada diária indica que a escala horária tinha muitos dados nulos agrupados em um mesmo período de um dia. 


\begin{table}[H]
\caption{Quantidade de registros nulos antes e após a conversão para dados diários das estações meteorológicas automáticas do INMET.}
\label{tab:estacoes_convencionais_inmet_dados_nulos}
\begin{adjustbox}{width=\textwidth}
\begin{tabular}{|l|c|r|r|r|}
\hline
\textbf{Variável} & \textbf{Registros nulos (Qtd.)} & \textbf{Registros nulos (\%)} & \textbf{Registros nulos diários (Qtd.)} & \textbf{Registros nulos diários (\%)} \\
\hline
Temperatura máxima  & 496.231 & 4,05\% &125.226 & 6,04\% \\
\hline
Temperatura mínima & 496.396 & 4,05\% & 125.226 & 6,04\% \\
\hline
Temperatura média & 493.102 & 4,02\% & 125.014 & 6,03\% \\
\hline
\end{tabular}
\end{adjustbox}
\end{table}

\section{Registros diários das estações automáticas do LabMet}

Por estarem na frequência diária, os dados das estações meteorológicas automáticas do LabMet não sofreram alterações. Verificando a ocorrência de valores nulos, não identificamos nenhum valor nulo na variável de temperatura miníma, um registro nulo na temperatura máxima, e dois registros nulos na variável da temperatura média. 

\section{Eliminando valores inconsistentes}

Nesta etapa verificamos a ocorrência de dados inconsistentes. Geralmente, essas inconsistências são geradas por erros de leituras dos sensores, como problemas de calibração ou defeitos. Baseado em \citeonline{baba2014correccao}, subdividimos as inconsistências em três grupos, inconsistências de limites, inconsistências lógicas, e inconsistências temporais. 

\begin{itemize}
	\item Inconsistência de limites: identifica valores que não respeitam os límites básicos da variável representada, ou seja, um valor fisicamente impossível de ser obtido. Nos dados de temperatura consideramos como inconsistência, para o Brasil, valores menores que -30ºC e maiores que 50ºC. Qualquer valor fora do intervalo estaria incorreto, pois representaria um valor impossível para a variável em questão;
	
	\item Inconsistências lógicas: quando temos dados de temperatura máxima, média e mínima referentes a um intervalo de tempo monitorado, estes dados devem respeitar sua condição lógica básica, ou seja, o valor de temperatura média para um período não pode ser maior que a temperatura máxima ou menor que a temperatura mínima deste mesmo período. Do contrário, esses valores serão considerados inconsistentes;     
	\item Inconsistências temporais: identifica valores inconsistentes baseado em seus valores históricos. Para esta análise, utilizamos a metodologia elaborada por \cite{mateo2008design}, em que consiste em determinar a consistência do dado avaliado baseado no valor do dia anterior, no valor do dia  seguinte e nos dados referentes aos três dias (anterior, corrente, posterior) do ano anterior. Se o valor do dado analisado se afastar da média dos valores analisados 3 vezes o valor do desvio padrão, este valor pode indicar problemas de leituras.  
\end{itemize}

Ao final desta etapa, utilizamos os dados de temperatura máxima e mínima para validar o dado de temperatura média, mas após esta etapa eliminamos os dados de temperatura máxima e mínima e passamos a trabalhar apenas com a temperatura média. Antes de eliminar os valores inconsistentes, nosso conjunto de dados, resultado da combinação de todas as bases de dados, continha 6.300.381 registros no total, com 619.192 destes com valores nulos, ou seja, 9,82\% de todos os registros. Após a eliminação dos valores inconsistentes, a quantidade de registros nulos subiu para 958.992, representando 15,22\% do dado. 

\renewcommand{\cleardoublepage}{}
\renewcommand{\clearpage}{}
\vspace{5mm}