\chapter{Introdução}

\section{Contextualização}
As atividades de previsão desempenham um papel fundamental em nossas vidas. Todos os dias, a previsão do tempo nós informa como estará o tempo no dia seguinte, na semana seguinte, e até no mês seguinte. A temperatura, sendo um dos mais importantes parâmetros que são apresentados em previsões do tempo, tem um impacto direto na evaporação, derretimento de neve, geada e um impacto indireto nas condições atmosféricas e precipitação \cite{hansen2006global}. De acordo com recentes estudos sobre os impactos das mudanças climáticas, agricultura, vegetação, recursos hídricos e o turismo são os setores mais afetados diretamente por mudanças de temperatura. Portanto, é necessário prever a temperatura com precisão para evitar perigos inesperados causados pela variação da temperatura, como geadas e secas que podem causar danos financeiros e perdas humanas \cite{kaymaz2005hazards}.

\section{O problema proposto}
Diante desse contexto, este trabalho tem como objetivo prever o comportamento da temperatura média do ar para um intervalo de um ano utilizando séries temporais de temperatura obtidas de estações meteorológicas distribuídas por todo o território brasileiro.

Para facilitar o entendimento do problema e da solução a ser proposta, utilizamos a técnica do 5W's, que consiste em responder as seguintes perguntas:

\textbf{Why?}: Variações de temperatura podem ter impacto direto na produção agrícola, geração de energia, turismo e até na saúde da população, por isso, prever a temperatura com precisão é essencial para evitarmos esses riscos.

\textbf{Who?}: Os dados foram coletados das estações convencionais e automáticas do Instituto Nacional de Meteorologia do Brasil (INMET) e do Laboratório de Meteorologia da Universidade Federal do Vale do São Francisco (LabMet).

\textbf{What?}: Prever o comportamento da temperatura média do ar para um intervalo de um ano utilizando séries temporais de temperatura obtidas de estações mateológicas espalhadas por todo o território brasileiro.

\textbf{Where?}: Estações meteorológicas espalhadas por todo o território brasileiro.  

\textbf{When?}: O conjunto de dados das estações convencionais do INMET contém observações do período de 1961 à 2019. Já as estações automáticas, também do INMET, que começaram a ser implantadas no Brasil a partir no inicio deste século, possui dados de 2000 à 2019. Por último, as estações automáticas do LabMet possui dados de 2007 à 2019. 