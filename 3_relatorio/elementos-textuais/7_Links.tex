\chapter{Links}

Disponibilizamos os conjuntos de dados obtidos na etapa da coleta dos dados como \textit{Datasets} na plataforma Kaggle através dos links: 

\begin{itemize}
  \item Brazil Weather, Conventional Stations (1961-2019):
 
  \href{https://www.kaggle.com/saraivaufc/conventional-weather-stations-brazil}{https://www.kaggle.com/saraivaufc/conventional-weather-stations-brazil}
  
  \item Brazil Weather, Automatic Stations (2000-2019):
 
  \href{https://www.kaggle.com/saraivaufc/automatic-weather-stations-brazil}{https://www.kaggle.com/saraivaufc/automatic-weather-stations-brazil}
  
  \item LabMet - Automatic Weather Stations (2007-2019):
 
  \href{https://www.kaggle.com/saraivaufc/automatic-weather-stations-labmet}{https://www.kaggle.com/saraivaufc/automatic-weather-stations-labmet}
\end{itemize}

Todos os scripts desenvolvidos, os parâmetros ARIMA ajustado para as 88 estações avaliadas e o modelo LSTM treinado estão disponíveis na plataforma Github através do link: \href{https://github.com/saraivaufc/TCC-Ciencia-de-Dados}{https://github.com/saraivaufc/TCC-Ciencia-de-Dados}

O vídeo com uma breve explanação sobre este trabalho está disponível na plataforma YouTube, acessível através do link: 
 
Link para o vídeo: youtube.com/... 

Link para o repositório: github.com/... 